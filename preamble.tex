\usepackage[utf8]{inputenc}
\usepackage[T2A]{fontenc}
\usepackage[russian]{babel}
\usepackage{amsmath,amssymb,amsthm} % математические пакеты
\usepackage{geometry}               % для установки полей
\geometry{a4paper, margin=2cm}      % поля по 2 см, формат А4
\usepackage{graphicx}               % для вставки картинок
\usepackage[hidelinks]{hyperref}    % для гиперссылок и кликабельного оглавления
\usepackage{indentfirst}            % первый абзац в разделе с красной строки
\usepackage{float}                  % положение картинок
\usepackage{booktabs}               % таблица
\usepackage{multirow}
\usepackage{tikz}
\usepackage{quiver}


% Inkscape
\usepackage{import}
\usepackage{xifthen}
\usepackage{pdfpages}
\usepackage{transparent}

\newcommand{\incfig}[1]{
    \def\svgwidth{\columnwidth}
    \import{./Figures/}{#1.pdf_tex}
}


% Увеличение зазора между строками матрицы
\renewcommand{\arraystretch}{1.2}

\theoremstyle{plain}
\newtheorem{theorem}{Теорема}[section] 
% "theorem" – имя окружения
% "Теорема" – заголовок по умолчанию
% [section] означает, что нумерация будет идти по секциям (например, "Теорема 1.2").

\newtheorem{lemma}[theorem]{Лемма}
% Лемма будет нумероваться вместе с теоремами, но иметь свой собственный заголовок

\theoremstyle{definition}
\newtheorem{definition}[theorem]{Определение}
% "definition" – имя окружения
% "Определение" – заголовок
% Нумерация идёт подряд с теоремами

\theoremstyle{remark}
\newtheorem{remark}[theorem]{Замечание}

% Дополнительные настройки пакетов, стили заголовков и т. д.
% \newtheorem{theorem}{Теорема}[section]
% \newcommand{\R}{\mathbb{R}} % Пример своего макроса для мн-ва действительных чисел
